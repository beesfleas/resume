
\documentclass[letterpaper,11pt]{article}
\newlength{\outerbordwidth}
\pagestyle{empty}
\raggedbottom
\raggedright
\usepackage[svgnames]{xcolor}
\usepackage{framed}
\usepackage{tocloft}
\usepackage{array}
\usepackage{hyperref}
\input{glyphtounicode}

%-----------------------------------------------------------
% Edit these values as you see fit

\setlength{\outerbordwidth}{1pt}  % Width of border outside of title bars
\definecolor{shadecolor}{gray}{0.4}  % Outer background color of title bars (0 = black, 1 = white)
\definecolor{shadecolorB}{gray}{0.96}  % Inner background color of title bars

%-----------------------------------------------------------
% Margin setup

\setlength{\evensidemargin}{-0.25in}
\setlength{\headheight}{-0.5in}
\setlength{\headsep}{0in}
\setlength{\oddsidemargin}{-0.25in}
\setlength{\paperheight}{11in}
\setlength{\paperwidth}{8.5in}
\setlength{\tabcolsep}{0in}
\setlength{\textheight}{9.5in}
\setlength{\textwidth}{7in}
\setlength{\topmargin}{-0.3in}
\setlength{\topskip}{0in}
\setlength{\voffset}{0.1in}

\linespread{0.96} % determines line spacing (1.0 is normal, 1.3 is 1.5x, 1.6 is 2.0x, etc.)

%-----------------------------------------------------------
% Custom commands
\newcommand{\resitem}[1]{\item #1 \vspace{-8pt}}
\newcommand{\resheading}[1]{\vspace{8pt}{\Large \textbf{#1}}\\\vspace{-8pt}\hrulefill}
\newcommand{\ressubheading}[4]{\vspace{3pt}
\begin{tabular*}{7.0in}{l@{\cftdotfill{\cftsecdotsep}\extracolsep{\fill}}r}
		\textbf{#1} & #2 \\
		\textit{#3} & \textit{#4} \\
\end{tabular*}\vspace{-6pt}}
\newcommand{\ressubheadingsmol}[2]{\vspace{1pt}
\begin{tabular*}{7.0in}{l@{\cftdotfill{\cftsecdotsep}\extracolsep{\fill}}r}
		\textbf{#1} & \textit{#2} \\
\end{tabular*}\vspace{-6pt}}
\begin{document}    
\begin{center}
    \textbf{\LARGE \scshape Inesh Chakrabarti} \\ \vspace{1pt}
    \small 858-925-3059 $|$ \href{mailto:inesh33@g.ucla.edu}{\underline{inesh33@g.ucla.edu}} $|$ 
    \href{https://www.linkedin.com/in/inesh-chakrabarti-878602183}{\underline{linkedin.com/in/inesh-chakrabarti}} $|$
    \href{https://github.com/beesfleas}{\underline{github.com/beesfleas}}
\end{center}

%%%%%%%%%%%%%%%%%%%%%%%%%%%%%%
\resheading{Education}
%%%%%%%%%%%%%%%%%%%%%%%%%%%%%%
\ressubheading{University of California, Los Angeles}{Los Angeles, CA}{B.S. Electrical Engineering, M.S. Electrical Engineering}{Expected Graduation -- Jan 2027}
\begin{itemize}
	\resitem{\textbf{Graduate GPA:} 4.0 / 4.0, \textbf{Undergraduate Major GPA:} 3.8 / 4.0}
	\resitem{\textbf{Coursework:} Large Scale Data Mining, Convex Optimization, Deep Learning, Software Engineering, Embedded Systems, Computer Architecture, GPU Microarchitectures, Numerical Computing, Stochastic Systems,  Communications, Signals and Systems, Probability and Statistics}
    \resitem{\textbf{Leadership/Societies:} American Nuclear Society (President, Founder), Eta Kappa Nu (Mentorship Chair)}
\end{itemize}

%%%%%%%%%%%%%%%%%%%%%%%%%%%%%%
\resheading{Skills}
%%%%%%%%%%%%%%%%%%%%%%%%%%%%%%
\begin{itemize}
\resitem{\textbf{Programming Languages:} C, C++, Python (NUMBA, PySpark, Matplotlib, PyTorch, Pandas, Keras, Tensorflow), Triton, SQL, x64, C\#, Java, MATLAB, R, JavaScript}
\resitem{\textbf{Tools:} Docker, Git, LangGraph, MongoDB, LTSpice, GDB, Unix Shell, CUDA, OpenMP, Joblib, Django, NVIDIA Nsight Compute, Apache Spark, Fuzzing (AFL), CI/CD}
\end{itemize}

\resheading{Experience}
%%%%%%%%%%%%%%%%%%%%%%%%%%%%%%
\ressubheadingsmol{UCLA Lin Yang Research Group (AI Researcher)}{Febuary 2025 - Present}{}
\begin{itemize}
  \resitem{\href{https://arxiv.org/abs/2504.14569}{\textit{NoWag: A Unified Framework for Shape Preserving Compression of Large Language Models} 
    } Lawrence Liu, \textbf{Inesh Chakrabarti}, Yixiao Li, Mengdi Wang, Tuo Zhao, Lin F. Yang \\ Publication accepted to \textbf{COLM} and \textbf{ICLR SLLM Workshop}}
    \resitem{Built dequantization/inference kernels in C (CUDA) for parallelization over multiple GPUs for over 10x speedup while using \textbf{48x less calibration data }and matching SOTA VQ method performance}
   \resitem{Implemented Trellis Quantization and benchmarking in Python for NoWag, a set of shape-preserving pruning and quantization algorithms for LLMs}
\end{itemize}

\ressubheadingsmol{UCLA Complex Networks Group (Paid Machine Learning Researcher)}{February 2022 - June 2024}
\begin{itemize}
% • Developed software modules in Python for data pre-processing, visualization, and post-processing
% • Constructed a speech-to-text pipeline that subtitles recall experiments with precise temporal acc.
% • Led a group of undergraduate students to conduct data annotation and processing
    \resitem{Implemented High Frequency Oscillation Detector using Variational Autoencoder for neural signals, \textbf{doubling} number of detections with only a \textbf{10\%} increase in false positives}
    \resitem{Constructed a speech to text pipeline that subtitled recall experiments with precise temporal acc.}
    \resitem{Processed and visualized neural spike data using Python and MATLAB to demonstrate correlation between individual neural spikes and character recognition from animation}
    \resitem{Developed a complete pipeline for EEG data analysis with wavelet transform pre-processing to predict human movement using transformer, LSTM, and CNN models.}

\end{itemize}


%%%%%%%%%%%%%%%%%%%%%%%%%%%%%%
\resheading{Projects}
%%%%%%%%%%%%%%%%%%%%%%%%%%%%%%
\ressubheadingsmol{Database Benchmarking Tool}{September 2025 - December 2025}
\begin{itemize}
    \resitem{Engineered a\textbf{ novel benchmarking tool} by translating TPC-DS SQL queries into \textbf{PySpark} via Abstract Syntax Tree (AST) manipulation and injecting realistic User-Defined Functions (UDFs).}
    \resitem{Scraped and analyzed public PySpark workflows from GitHub to create a data-driven model of modern data pipelines, guiding the synthesis of UDFs based on metrics like cyclomatic complexity.}
\end{itemize}
\ressubheadingsmol{Reinforcement Learning Hearts}{September 2024 - January 2025}
\begin{itemize}
    \resitem{Created RL agent for Hearts using \textbf{Counterfactual Regret Minimization} and \textbf{Monte Carlo Tree Search} that reaches approximate Nash Equilibrium.}
    \resitem{Enhanced the Hearts project with a Tkinter UI and collaborated in a 3-person team, providing a \textbf{real-time interface} allowing for physical gameplay simulation via computer vision.}
\end{itemize}


\ressubheadingsmol{Large Scale Data Mining}{January 2023 - March 2023}

\begin{itemize}
    \resitem{Experimented with  different forms of clustering for text and image data including \textbf{Kmeans} and  \textbf{HDBSCAN clustering}}
    \resitem{Constructed an end to end pipeline for news classification using grid-searching over different vectorization and classification models including BOW, TFIDF, SVM, Naive Bayes, etc. while leveraging \textbf{cuML} and \textbf{cuDF} for GPU optimization.}
\end{itemize}
% https://github.com/LawrenceRLiu/ACM-advanced-track
% \ressubheadingsmol{Habit-Tracking IoT Tamagotchi with Web App}{Jan 2022}
% \begin{itemize}
% \resitem{Built backend of web app with MongoDB  with login authentication logging users' schedules with an IoT device registering task completion.}
% \resitem{Programmed ESP32 to update device's expressions in real-time based on schedule adherence.}
% \end{itemize}

%%%%%%%%%%%%%%%%%%%%%%%%%%%%%%

\end{document} 
